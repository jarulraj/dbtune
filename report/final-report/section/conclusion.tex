\section{Conclusion} \label{sec:conclusion}

DBMS performance tuning is a niche skill that requires much experience
and experimentation to get right. Machine learning techniques can help
automate this process by using prior knowledge to estimate the
performance of the DBMS on a particular SQL workload
without having to actually execute it.
To achieve this goal, we have taken a two-step approach where
we first map a workload to a well-known benchmark.
Armed with this information, we then use a benchmark-specific
regression estimator to obtain performance estimates
under different DBMS configurations.

Using decision trees, we find that we can classify workloads
using a few defining features.
Our experiments show that a CART-based decision tree classifier
achieves 91\% accuracy on the classification problem.
Thus, the decision trees learned to effectively discriminate
between the different benchmark classes.
We use Gaussian Process Regression estimators for predicting
the throughput and latency of the DBMS while executing a
SQL workload under a particular configuration.
We demonstrate that these estimators achieve a median R-squared score
of 0.98 for estimating latency and 0.95 for estimating throughput.
Furthermore, these highly accurate estimators can be trained
with as few as 400 samples for latency and 900 samples for throughput.
Our analysis using Lasso Regression estimators provided key
insights into the features that are most influential in
determining throughput and latency.

In our experiments, we generate synthetic variants of the
benchmarks in OLTPBench to emulate real-world workloads as
we felt that it is a good starting point for evaluating the
viability of our techniques.
We plan to continue our evaluation using a larger real-world
dataset that includes more realistic samples.
We note that we leave the problem of using the performance
estimator to find an optimal DBMS configuration for a given
workload for future work. The estimator allows us to obtain
inexpensive estimates without actually running the workload
under a specific configuration.
In addition, we will extend our auto-tuning framework to
support other DBMSs like MySQL, and more hardware profiles.
We anticipate that this larger and more realistic dataset
will help us obtain additional insights about how best to
tune the DBMS configuration to maximize performance on
arbitrary SQL workloads.
