\section{Conclusion} \label{sec:conclusion}

DBMS performance tuning is a niche skill that requires much experience
and experimentation to get correct. However, machine learning
techniques can help automate this process by using prior knowledge to
estimate performance without having to go through a time-consuming
experiment. To achieve this goal, we have taken a two-step approach
where we first use map a workload to a well-studied benchmark. Armed with
this information, we then use a regression estimator to estimate the
performance of the benchmark under a given environment.

Using decision trees to map the workloads, we find that we can
classify workloads using a few defining characteristics. The decision
trees produces effectively discriminate between the different
benchmarks and are very intuitive. For performance estimation, we trained estimators
using Gaussian Process Regression that show very accurate estimation
of database throughput and latency. In addition, these highly accurate
estimators can be trained with as few as 600 samples! Further analysis
of performance estimation using Lasso Regression provided key insights
into the features that are most influential in determining throughput
and latency.

While promising, this work only uses synthetic data generated from
OLTPBench. We hope to continue this work using a larger dataset that
includes mixtures of benchmarks along with real-world workloads. In
addition, we hope to extend our data to include performance
estimation across different hardware profiles and DBMS's. This larger
amount of more realistic data should help us get additional insights
about how best to estimate performance of arbitrary database workloads.