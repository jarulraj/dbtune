\section{Concurrency Control in Key-Value Stores} \label{sec:tm}

The literature on practical tests of HTM is very limited. Key-value stores
present an appealing test case for the technology as a concurrency control (CC)
mechanism.

\subsection{Key-Value Store}

A key-value store presents an associative data access interface that maps a set
of keys $K$ to a set of values $V$, where the keys and values can be arbitrary
objects. The keys are generally prehashed to obtain an uniform distribution. The
values can be stored separately and fixed-length pointers can be used to
represent them in the key-value store implementation.  

The main operations provided in the key-value store's interface are
\textsc{GET}(key), \textsc{PUT}(key,value), and \textsc{DELETE}(key). Readers
can conflict with writers and writers can conflict with other writers. Thus, a
concurrency control mechanism is required to provide isolation between
conflicting accesses. Furthermore, clients of the key-value store often require
that reads/writes to more than one entry in the store appear
atomic. Synchronizing this type of transaction -- a \textit{multi-key
transaction} -- is the focus of this project.

\subsection{Optimistic Concurrency Control in Haswell}

A simple mechanism that can be used to provide isolation is coarse-grained
locks. This allows at least a limited amount of safe concurrency.  However,
this is a pessimistic concurrency control protocol that targets high-contention
workloads. Read accesses also need to incur the locking overhead in this scheme.

For low contention workloads like read-dominant workloads, an optimistic
concurrency control scheme generally provides better performance, as the
validation is postponed to the end of the transaction. The concurrency control
protocol is essentially not exposed to the user, which allows non-conflicting
accesses to avoid the locking overhead.  Optimistic concurrency control
mechanisms still rely on locking, but they move this functionality to the
language runtime layer in the case of software transactional memory schemes
(STM) or to the hardware layer in the case of hardware transaction memory
schemes (HTM). The programmer is presented only the higher-level abstraction of
transactions. We focus on HTM in this project.

HTM simplifies concurrent programming by providing support for atomic execution
of a set of load and store instructions by using transactional caches. Intel TSX
provides an HTM implementation. It allows non-conflicting transactions to elide
locks entirely, serializing only conflicting transactions.

Conflict resolution mechanism is flexible in the RTM interface. The programmer
can define the fallback mechanism to resolve conflicts. For instance, a group
sum operation using RTM interface in Intel Haswell is shown in \Cref{fig:rtm}.
In contrast, the HLE interface relies on the processor to handle conflicts -- a
more conservative policy. We plan to evaluate both interfaces in our project.

\lstset{basicstyle=\ttfamily\fontsize{9}{10}\selectfont,
morekeywords={if,else,end}, numbers=left} \begin{figure}
    \parbox[t]{0.45\textwidth}{\lstinputlisting{figure/rtm.c}} \caption{Group
        sum operation using RTM interface. A mutex is obtained in fallback path
to avoid data race with the normal transaction execution path.} \label{fig:rtm}
\end{figure}
