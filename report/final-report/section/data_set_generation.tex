\section{Data Set} \label{sec:data_set}

The dataset required for this project would ideally comprise of a 
collection of real world SQL workloads, DBMS configurations and performance
metrics. Collecting and curating such a dataset is itself an interesting
problem.
However, in this project, we first want to experiment with a smaller dataset
to better understand the features relevant for our learning problem.
Therefore, we chose to generate the dataset using
OLTP-Bench~\citep{oltpbench14}, an extensible DBMS benchmarking framework.
We use the SQL workloads of standard benchmarks already available
in OLTPBench. 

We generate more synthetic variants of these workloads for training and
testing purposes. As we mentioned earlier, while this synthetic data will not be
representative of real-world workloads, we feel it is a good starting point for
evaluating viability of the approach outlined above.
We extract several features from the workload such as types of database queries,
distribution of query types, table access patterns, etc. using 
a workload analyzer. We hook this analyzer into Postgres~\citep{postgres91} to collect 
features from the DBMS.

The key reasons for why we use this framework to generate the dataset 
are the following:

\begin{itemize}
  \item It supports several relational DBMSs through the JDBC interface
  including Postgres, MySQL, and Oracle.
  \item It allows us to control the workload mixture in a benchmark. For
  instance, we can adjust the percent of read and update transactions in 
  the YCSB~\citep{ycsb} benchmark to generate different variants of the workload.
  \item It supports user-defined configuration of the rate at which the
  transaction requests are submitted to the DBMS. This allows us to emulate
  different world workloads with varying degrees of concurrency.
  \item The framework exposes statistics that are complementary to the 
  internal statistics of the DBMS~\citep{postgres14}. We extract features from
  these statistics.
\end{itemize}

We implemented a dataset generator that runs different benchmarks supported
by OLTP-Bench on a Postgres DBMS. After every workload execution, we
collect statistics from the DBMS as well as from the testbed. We alter the
workload mixture in all the benchmarks to generate different variants and
emulate real world workloads. The key characteristics of the benchmarks
that we use for generating the dataset are presented in \cref{tab:benchmarks}.

\begin{table*}
\centering
\small{
  \centering
  \begin{tabular}{lllll lllll} \toprule
   Workloads & Tables & Columns & Pr. & Indexes &	Fr. & 
   Txn. & \# of & Application  & Attributes  \\
   & & & Keys & & Keys & Types & Joins  & domain & \\
   \midrule
	AuctionMark 	& 16 	& 125  &	16 &	14 &	41 &	10 &	10 &	Online &
	Non-deterministic\\
	& & & & & & & & Auctions &  heavy transactions \\
	%Epinions 	5 	21 	2 	10 	0 	9 	3 	Social Networking 	Joins over many-to-many%relationships 
	%JPAB 	7 	68 	6 	5 	3 	4 	N/A 	Object-Relational Mapping 	Bursts of random%reads, pointer chasing 
	%ResourceStresser 	4 	23 	4 	0 	0 	6 	2 	Isolated Resource Stresser 	CPU-,disk-, lock-heavy transactions 
	SEATS &	10 &	189 &	9 	& 5 	& 12 	& 6 &	6 &	On-line Airline  &
	Secondary indices queries\\
	& & & & & & & & Ticketing & foreign-key joins \\
	TATP &	4 	& 51 &	4 &	5 &	3 &	7 &	1 &	Caller Location  &	Short, read-mostly\\
	& & & & & & & & App & non-conflicting\\
	& & & & & & & & &  transactions\\
	TPC-C &	9 &	92 &	8 &	3 &	24 &	5 &	2 &	Order Processing & Write-heavy
	transactions\\
	Twitter &	5 &	18 &	5 &	4 &	0 &	5 &	0 &	Social Networking & Client-side joins \\
	& & & & & & & & & on	graph data \\
	%Wikipedia 	12 	122 	12 	40 	0 	5 	2 	On-line Encyclopedia 	Complex trans,
	% large data, skew 
	YCSB &	1 &	11 &	1 &	0 &	0 &	6 &	0 &	Scalable NoSQL store &	Key-value queries \\
   \bottomrule
   \end{tabular}
 }
\caption{Key characteristics of the benchmarks used in our evaluation. ``Pr. key''
denotes primary key and ``Fr. key'' denotes foreign key.}
\label{tab:benchmarks}
\end{table*}


