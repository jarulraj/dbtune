\section{Project Plan} \label{sec:plan}

Broadly speaking, we plan to:
\begin{itemize}
\item Implement two different TSX-based concurrency control protocols for
  multi-key transactions on a key-value store
\item Evaluate the performance of those schemes under different workloads
\item Compare them against pessimistic CC protocols
\end{itemize}
The goals of the project are as follows:
\begin{itemize}
\item \textbf{75\% goal:} Implement concurrency control protocol for a key-value
        store using HLE and RTM, and compare the performance of these two
        protocols.
\item \textbf{100\% goal:} Additionally, compare the above protocols with
        traditional pessimistic concurrency control protocols, specifically
        spin-locks and a lock manager. This will demonstrate the advantages of
        using HTM for this synchronization problem, or else show that 
        HTM does not help in practice.
\item \textbf{125\% goal:} Modify the HTM-based protocols to handle dynamic
        read/write sets. Alternatively, implement a software optimistic concurrency
        control protocol (like timestamp order protocol) and compare it against the
        aforementioned protocols. This would serve as a basic check that it is in
        fact the hardware, not the approach, that is the cause of any differences in
        performance.
\end{itemize}

\subsection{Resources Required}
The main resource necessary for this project is access to a machine
with a Haswell processor. We have already received access to this
resource with Dong's help.

It would be useful to have access to the existing codebase used by Prof.\ Andersen's lab to run similar HTM experiments. In particular, it would be helpful to have access to the key-value store implementation. We are expecting that Dong will be able to give us access to this, as well.

\subsection{Experiments}
Our experiments are inspired by those reported in \citep{tran2010}. We will
restrict ourselves to a small, fixed number of key-value store entries. In each
experiment, we will measure the time it takes to run a randomly generated
workload of datastore operations, given a particular concurrency control
scheme. Each workload will simply consist of looking up some set of keys and
trivially modifying their values (e.g., incrementing). Specifically, we will run
the following experiments for each type of CC mechanism:

\begin{enumerate}
\item With several fixed sizes for read/write sets and numbers of threads, vary the contention level between operations on different threads. This will allow us to determine how each of the CC mechanisms scales with respect to contention.
\item With several different fixed contention levels and numbers of threads, vary the size of the read/write sets. This will allow us to determine how each of the CC mechanisms scales with respect to the read/write set. (For HTM, this may be a very important factor, since transactions abort based on conflicts anywhere in the read/write set.)
\item With several different fixed contention levels and read/write set sizes,
    vary the number of threads running. This will allow us to determine how much benefit each mechanism is able to benefit from adding more parallelism.
\end{enumerate}

\subsection{Steps for Execution}
The required steps for executing this project are as follows:
\begin{enumerate}
\item Familiarize ourselves with an existing basic key-value store, such as that
  built by Prof.\ Andersen's group, or (if that turns out to be impractical)
  implement a simpler key-value store. In the latter case, we can
  minimize implementation effort by keeping the data structures very simple.
\item Design the structure of the transaction manager to support multiple CC mechanisms.
\item Using the Haswell TSX APIs, implement a transaction manager that optimistically attempts to execute a transaction, and retries according to either an HLE strategy or a custom RTM-specified strategy.
\item Build a system to generate test workloads with different amounts of contention. It should also allow specifying thread assignments if necessary.
\item Run the experiments for the two HTM approaches. Some tweaking will be necessary to find the most informative thread numbers, read/write set sizes, and contention levels.
\item Implement spin-locks and a lock manager.
\item Rerun the experiments for the pessimistic concurrency control approaches.
\item
  \begin{enumerate}
  \item Modify protocols for dynamic read/write sets and rerun the
    experiment using modified protocols, or
  \item Implement software-based OCC and rerun the experiments for OCC.
  \end{enumerate}
\item Collate/visualize data and write up report.
\end{enumerate}

\subsection{Progress}
As of this report's writing, we have made significant progress towards 
achieving our 100\% goal. In particular, we have implemented the following
pieces:
\begin{itemize}
\item \textbf{Key value store:} Implemented by Joy in C++, the key value store 
uses a hash table as its underlying data structure, relying on the open source
MurmurHash algorithm to hash keys. The kv store supports arbitrary keys and 
values as well as convenience methods for 64-bit integer keys and string 
values, which will be used by the testing framework. By implementing the hash 
buckets as linked lists, the kv store is safe in the face of concurrent 
insertions to the same bucket.
\item \textbf{Pessimistic locking schemes:} Implemented by Thomas, we currently 
support both spin locks and a lock table with reader/writer locks, as described 
in \Cref{sec:pessimistic}.
\item \textbf{Testing framework:} Implemented by Jesse, the testing framework 
currently acts as a sanity check to ensure that our concurrency control schemes 
are implemented correctly. It accomplishes this by spawning two threads, each 
of which repeatedly runs one transaction - a transaction that writes a random 
value to a given key, and a transaction that repeatedly reads the value of a 
single key. The reader thread then checks that the value that it read did not 
change between reads of the same transaction, ensuring that the reader and 
writer transactions were not allowed to operate on the same key concurrently.
\end{itemize}

\subsection{Completion Timeline}
Now that the fundamental pieces are in place in terms of the key value store, 
testing framework, transaction manager, and interaction between them, we have 
a few tasks left to complete before writing our final report.
\begin{itemize}
\item \textbf{HTM transaction manager:} Probably the most important remaining 
piece of the project, the HTM transaction manager will be implemented by Joy. 
We have already begun to look into the instructions necessary to make this 
work, so implementing it should be fairly straight forward, and we anticipate 
completing this by 11/16.
\item \textbf{Testing framework:} The next most important thing will be to 
flesh out the testing framework, providing support for varying the number of 
threads, level of contention in the workload, and size of the transaction's 
working sets. This will be implemented by Jesse, and should also be finished by 
11/16.
\item \textbf{Data gathering:} The final component necessary for us to reach 
our 100\% goal is to run a large number of experiments using the testing 
framework, record the data, and generate graphs as appropriate for the paper. 
A more extensive discussion of this is included in \Cref{sec:eval}. We will 
also work to refine the various CC schemes here to achieve the best performance 
possible for each. This task is assigned to Thomas, and should be completed by 
11/23.
\item \textbf{Stretch goal:} Once we have completed our 100\% goals, we will 
move on to our stretch goals. If the results of our testing indicate that HTM 
is significantly faster than the pessimistic schemes, we intend to implement 
software based OCC to determine whether the speed up is the result of the use 
of optimistic concurency control or the use of hardware based control. 
Otherwise, we will explore the effects of dynamic read/write sets on the 
performance of the different CC schemes. We will cut off work on the stretch 
goals by 12/6 to ensure we have enough remaining time to complete the final 
paper.
\item \textbf{Final Paper:} Each member of the group will be responsible for 
the sections of the paper that relate to the aspects of the project they 
implemented. We intend to finish the paper 5 minutes before the deadline.
\end{itemize}

