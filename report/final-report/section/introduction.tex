\section{Introduction} \label{sec:intro}

\subsection{Motivation}

DBMS tuning is a niche skill that involves configuring the DBMS suitably for the
underlying hardware as well as guiding the physical design of the database.
Doing these tasks effectively requires a deep knowledge of the SQL workload to
be run on the DBMS, extensive prior experience in DBMS tuning, as well as
significant amounts of time-consuming experimentation with candidate
configurations. This issue is prevalent across several widely used database
management systems. For instance, Robert Hass, a core developer of PostgreSQL, 
stated that \citep{hass12} :

\begin{displayquote}
``On one test, involving 32 concurrent clients, I found that wal\_buffers = 64
MB doubled performance as compared with wal\_buffers = 16 MB; however, on
another system, I found that no setting I tried produced more than a 10 \%
improvement over the auto-tuning formula.'' 
\end{displayquote}

We therefore propose applying machine learning methods to automate the process
of tuning the DBMS for a particular workload on a specific machine.

\subsection{Problem definition}

We address two problems in this project.
First, we try to \textit{map} a given workload comprised of
a set of SQL transactions to a standard database benchmark workload. 
This problem is independent of the underlying DBMS or hardware configuration.
This will allow us to use prior knowledge about the standard 
benchmark gained from previous DBMS deployments. 

We collect features that characterize the SQL workload as well as 
DBMS statistics.
We then use unsupervised techniques like clustering
and supervised techniques like decision-trees 
for mapping the workload.
Performance analysis of the resulting classifier is done via
cross-validation.

Second, we plan to \textit{estimate} the DBMS performance given a 
DBMS configuration, hardware setup and SQL workload. 
This is done with supervised techniques like 
Gaussian process regression.
We use the lasso regression estimator to identify key features 
that influence the throughput and average transactional latency 
of the DBMS.
Performance analysis of the estimator is also done using 
cross-validation.
We describe our progress on solving these problems in this report.

The goals of this project are the following : 
\begin{itemize}
  	\item to create a workload mapper that maps an arbitrary SQL workload to a
well-known standard benchmark
	\item to estimate the performance metrics of a DBMS given a workload and
configuration pair
\end{itemize}

We first describe how we generate the dataset in \cref{sec:data_set}. 
Then, we focus on the feature extraction problem in \cref{sec:features}.
We describe our experimental setup and dataset information in
\cref{sec:eval}. The evaluation results obtained for the classification
problem and the estimation problem are shown in \cref{sec:classfication}
and \cref{sec:estimation} respectively. Finally, we present the 
conclusions of this project in \cref{sec:conclusion}.