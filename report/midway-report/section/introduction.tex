\section{Introduction} \label{sec:intro}

\subsubsection*{Milestones:} 

First, we plan to \textit{map} a given workload comprised of 
SQL transactions to a standard database benchmark workload. This will 
allow us to use prior knowledge about the standard benchmark gained from
previous DBMS deployments. We plan to collect features like
workload characteristics and DBMS performance metrics. 
We may need to make use of a feature extraction algorithm like 
\textit{principal component analysis} (PCA) as a preprocessing step.
It will be interesting to see if feature extraction yields insights 
about the most influential features.
We will then
use unsupervised techniques like \textit{clustering} for mapping workloads. 
Performance analysis of the resulting classifier will be done via 
cross-validation.

Second, we plan to \textit{estimate} the DBMS performance given the DBMS 
configuration, workload and hardware setup. This will be done
with supervised techniques like \textit{Gaussian process regression}. 
As we expect a lot of features to be present in the input, we may 
need to make use of a feature extraction algorithm like 
principal component analysis (PCA) first. 
Performance analysis of the estimator will be done using cross-validation.

\subsubsection*{Dataset:} 

For the first part, we will use SQL workloads of standard benchmarks
in OLTPBench\citep{oltpbench14}, a well-known testbed for benchmarking 
relational databases, as training data. We plan to generate 
more synthetic variants of these workloads for training and testing
purposes. While this synthetic data will not be representative of 
real-world workloads, we feel it is a good starting point for 
evaluating viability of the approach outlined above.
We anticipate some preprocessing steps in order to extract
features from the workload such as types of database queries, 
distribution of query types, table access patterns, etc. This will 
involve making a fairly sophisticated workload analyzer. We will try 
to hook this into an existing database to collect the features.
For the second part, we plan to run different workloads in OLTPBench  
on the DBMS under different configurations and obtain performance metrics. 
This will provide both training and testing data.

\subsubsection*{Timeline:} 
We estimate that it will take 2 weeks for creating the workload analyzer
that extracts the required features. Then, we will spend 2 weeks 
solving the first problem as well as generating data for the second problem. 
We plan to solve the first problem by the midway report deadline.
After that, we plan to solve the second problem in another 2 weeks. 
We reserve the final 2 weeks for wrapping up the project report and poster.

\subsubsection*{Minimum goals:}
The minimum goals are : (a) to create a workload mapper that maps an arbitrary 
SQL workload to a well-known standard benchmark, and (b) to estimate the
performance of a DBMS given a workload and configuration pair.

\subsubsection*{Stretch goals:} 
%We hope to expand our work along two directions. 
As a real-world workload can exhibit some aspects of different standard 
benchmarks or can exhibit different characteristics across time (e.g. read-mostly 
during day and write-mostly during night), mapping the entire workload onto 
a single benchmark may not be an entirely accurate characterization. 
To help characterize the workload more accurately, we will consider mapping 
parts of a workload onto different benchmarks using techniques like  
\textit{multi-label prediction}. 

% Can we skip this for now ?
%Second, we can use \textit{Bayesian optimization} to find the 
%optimal configuration that can deliver highest performance given 
%information about the workload and hardware setup. We think that this is a 
%hard problem since the state space is large.

In this paper, we contribute a study of the effects of the Haswell HTM
implementation (RTM, specifically) on synchronous access to in-memory data
structures. Specifically, we focus on \textit{multi-key transactions} on a
\textit{key-value store}, comparing the performance of an HTM concurrency
control with pessimistic alternatives. In \Cref{sec:tm}, we describe the problem
of synchronizing multi-key transactions on a key-value store using hardware TM
support, and related work to date on this problem. We then present the
traditional pessimistic concurrency control protocols we use for comparison in
\Cref{sec:pessimistic}. We define the goals of this project, our progress so
far, and our plan to accomplish the remaining workload in \Cref{sec:plan}. Finally, we sketch our evaluation plan \Cref{sec:eval}.
